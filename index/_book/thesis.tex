% This is the Reed College LaTeX thesis template. Most of the work
% for the document class was done by Sam Noble (SN), as well as this
% template. Later comments etc. by Ben Salzberg (BTS). Additional
% restructuring and APA support by Jess Youngberg (JY).
% Your comments and suggestions are more than welcome; please email
% them to cus@reed.edu
%
% See http://web.reed.edu/cis/help/latex.html for help. There are a
% great bunch of help pages there, with notes on
% getting started, bibtex, etc. Go there and read it if you're not
% already familiar with LaTeX.
%
% Any line that starts with a percent symbol is a comment.
% They won't show up in the document, and are useful for notes
% to yourself and explaining commands.
% Commenting also removes a line from the document;
% very handy for troubleshooting problems. -BTS

% As far as I know, this follows the requirements laid out in
% the 2002-2003 Senior Handbook. Ask a librarian to check the
% document before binding. -SN

%%
%% Preamble
%%
% \documentclass{<something>} must begin each LaTeX document
\documentclass[12pt,twoside]{reedthesis}
% Packages are extensions to the basic LaTeX functions. Whatever you
% want to typeset, there is probably a package out there for it.
% Chemistry (chemtex), screenplays, you name it.
% Check out CTAN to see: http://www.ctan.org/
%%
\usepackage{graphicx,latexsym}
\usepackage{amsmath}
\usepackage{amssymb,amsthm}
\usepackage{longtable,booktabs,setspace}
\usepackage{chemarr} %% Useful for one reaction arrow, useless if you're not a chem major
\usepackage[hyphens]{url}
% Added by CII
\usepackage{hyperref}
\usepackage{lmodern}
\usepackage{float}
\floatplacement{figure}{H}
% End of CII addition
\usepackage{rotating}

% Next line commented out by CII
%%% \usepackage{natbib}
% Comment out the natbib line above and uncomment the following two lines to use the new
% biblatex-chicago style, for Chicago A. Also make some changes at the end where the
% bibliography is included.
%\usepackage{biblatex-chicago}
%\bibliography{thesis}


% Added by CII (Thanks, Hadley!)
% Use ref for internal links
\renewcommand{\hyperref}[2][???]{\autoref{#1}}
\def\chapterautorefname{Chapter}
\def\sectionautorefname{Section}
\def\subsectionautorefname{Subsection}
% End of CII addition

% Added by CII
\usepackage{caption}
\captionsetup{width=5in}
% End of CII addition

% \usepackage{times} % other fonts are available like times, bookman, charter, palatino

% Syntax highlighting #22
  \usepackage{color}
  \usepackage{fancyvrb}
  \newcommand{\VerbBar}{|}
  \newcommand{\VERB}{\Verb[commandchars=\\\{\}]}
  \DefineVerbatimEnvironment{Highlighting}{Verbatim}{commandchars=\\\{\}}
  % Add ',fontsize=\small' for more characters per line
  \usepackage{framed}
  \definecolor{shadecolor}{RGB}{248,248,248}
  \newenvironment{Shaded}{\begin{snugshade}}{\end{snugshade}}
  \newcommand{\KeywordTok}[1]{\textcolor[rgb]{0.13,0.29,0.53}{\textbf{{#1}}}}
  \newcommand{\DataTypeTok}[1]{\textcolor[rgb]{0.13,0.29,0.53}{{#1}}}
  \newcommand{\DecValTok}[1]{\textcolor[rgb]{0.00,0.00,0.81}{{#1}}}
  \newcommand{\BaseNTok}[1]{\textcolor[rgb]{0.00,0.00,0.81}{{#1}}}
  \newcommand{\FloatTok}[1]{\textcolor[rgb]{0.00,0.00,0.81}{{#1}}}
  \newcommand{\ConstantTok}[1]{\textcolor[rgb]{0.00,0.00,0.00}{{#1}}}
  \newcommand{\CharTok}[1]{\textcolor[rgb]{0.31,0.60,0.02}{{#1}}}
  \newcommand{\SpecialCharTok}[1]{\textcolor[rgb]{0.00,0.00,0.00}{{#1}}}
  \newcommand{\StringTok}[1]{\textcolor[rgb]{0.31,0.60,0.02}{{#1}}}
  \newcommand{\VerbatimStringTok}[1]{\textcolor[rgb]{0.31,0.60,0.02}{{#1}}}
  \newcommand{\SpecialStringTok}[1]{\textcolor[rgb]{0.31,0.60,0.02}{{#1}}}
  \newcommand{\ImportTok}[1]{{#1}}
  \newcommand{\CommentTok}[1]{\textcolor[rgb]{0.56,0.35,0.01}{\textit{{#1}}}}
  \newcommand{\DocumentationTok}[1]{\textcolor[rgb]{0.56,0.35,0.01}{\textbf{\textit{{#1}}}}}
  \newcommand{\AnnotationTok}[1]{\textcolor[rgb]{0.56,0.35,0.01}{\textbf{\textit{{#1}}}}}
  \newcommand{\CommentVarTok}[1]{\textcolor[rgb]{0.56,0.35,0.01}{\textbf{\textit{{#1}}}}}
  \newcommand{\OtherTok}[1]{\textcolor[rgb]{0.56,0.35,0.01}{{#1}}}
  \newcommand{\FunctionTok}[1]{\textcolor[rgb]{0.00,0.00,0.00}{{#1}}}
  \newcommand{\VariableTok}[1]{\textcolor[rgb]{0.00,0.00,0.00}{{#1}}}
  \newcommand{\ControlFlowTok}[1]{\textcolor[rgb]{0.13,0.29,0.53}{\textbf{{#1}}}}
  \newcommand{\OperatorTok}[1]{\textcolor[rgb]{0.81,0.36,0.00}{\textbf{{#1}}}}
  \newcommand{\BuiltInTok}[1]{{#1}}
  \newcommand{\ExtensionTok}[1]{{#1}}
  \newcommand{\PreprocessorTok}[1]{\textcolor[rgb]{0.56,0.35,0.01}{\textit{{#1}}}}
  \newcommand{\AttributeTok}[1]{\textcolor[rgb]{0.77,0.63,0.00}{{#1}}}
  \newcommand{\RegionMarkerTok}[1]{{#1}}
  \newcommand{\InformationTok}[1]{\textcolor[rgb]{0.56,0.35,0.01}{\textbf{\textit{{#1}}}}}
  \newcommand{\WarningTok}[1]{\textcolor[rgb]{0.56,0.35,0.01}{\textbf{\textit{{#1}}}}}
  \newcommand{\AlertTok}[1]{\textcolor[rgb]{0.94,0.16,0.16}{{#1}}}
  \newcommand{\ErrorTok}[1]{\textcolor[rgb]{0.64,0.00,0.00}{\textbf{{#1}}}}
  \newcommand{\NormalTok}[1]{{#1}}

% To pass between YAML and LaTeX the dollar signs are added by CII
\title{Forecasting Constituents of the MSCI Minimum Volatility Index Through
Logistic Regression}
\author{John A. Gilheany}
% The month and year that you submit your FINAL draft TO THE LIBRARY (May or December)
\date{November 6, 2017}
\division{Statistics}
\advisor{Professor Michael Parzen}
\institution{Harvard College}
\degree{Bachelor of Arts in Statistics (Honors)}
%If you have two advisors for some reason, you can use the following
% Uncommented out by CII
\altadvisor{David Kane}
% End of CII addition

%%% Remember to use the correct department!
\department{Statistics}
% if you're writing a thesis in an interdisciplinary major,
% uncomment the line below and change the text as appropriate.
% check the Senior Handbook if unsure.
%\thedivisionof{The Established Interdisciplinary Committee for}
% if you want the approval page to say "Approved for the Committee",
% uncomment the next line
%\approvedforthe{Committee}

% Added by CII
%%% Copied from knitr
%% maxwidth is the original width if it's less than linewidth
%% otherwise use linewidth (to make sure the graphics do not exceed the margin)
\makeatletter
\def\maxwidth{ %
  \ifdim\Gin@nat@width>\linewidth
    \linewidth
  \else
    \Gin@nat@width
  \fi
}
\makeatother

\renewcommand{\contentsname}{Table of Contents}
% End of CII addition

\setlength{\parskip}{0pt}

% Added by CII

\providecommand{\tightlist}{%
  \setlength{\itemsep}{0pt}\setlength{\parskip}{0pt}}

\Acknowledgements{
I want to thank Prof.~Parzen and David Kane for all of their help.
}

\Dedication{

}

\Preface{
This thesis explores a way of predicting index constituents using
logistic regression.
}

\Abstract{
The low-risk anomaly has created opportunities for arbitrage in the
financial markets. As Baker et al. discuss in ``Benchmarks as Limits to
Arbitrage: Understanding the Low-Volatility Anomaly,'' low-volatility
and low-beta portfolios outperform and high-volatility and high-beta
portfolios by a factor of several times due to benchmarking and
lottery-preferences. The iShares MSCI USA Minimum Volatility (USMV) is
an ETF tracking a minimum volatility index that was used to find data
and will be used for trading arbitrage. Frazzini et al. discuss
arbitrage opportunities by quantitative focused funds like AQR in
``Betting Against Beta'', and this thesis explores a more advanced type
of index front-running as a potential arbitrage opportunity. Data was
collected from USMV from its inception in October 2011, and from EUSA,
the parent ETF of USMV, from the same period until December 2016.
52-week trailing beta, 52-week trailing volatility, lagged price/book,
and current index membership were calculated, and a regression model was
run to quantify the relationship between current index membership and
these four variables. In the model, a probabilities of index membership
were calculated and an optimal cutoff was calculated to which the model
would be 95\% accurate of its findings of a stock to be in or out of
USMV, given the historical data. Backtesting with prior data showed with
a model accuracy of 95\%, arbitrage opportunities of X\% could be
collected after each rebalancing.
}

% End of CII addition
%%
%% End Preamble
%%
%

\usepackage{amsthm}
\newtheorem{theorem}{Theorem}[chapter]
\newtheorem{lemma}{Lemma}[chapter]
\theoremstyle{definition}
\newtheorem{definition}{Definition}[chapter]
\newtheorem{corollary}{Corollary}[chapter]
\newtheorem{proposition}{Proposition}[chapter]
\theoremstyle{definition}
\newtheorem{example}{Example}[chapter]
\theoremstyle{definition}
\newtheorem{exercise}{Exercise}[chapter]
\theoremstyle{remark}
\newtheorem*{remark}{Remark}
\newtheorem*{solution}{Solution}
\begin{document}

% Everything below added by CII
  \maketitle

\frontmatter % this stuff will be roman-numbered
\pagestyle{empty} % this removes page numbers from the frontmatter
  \begin{acknowledgements}
    I want to thank Prof.~Parzen and David Kane for all of their help.
  \end{acknowledgements}
  \begin{preface}
    This thesis explores a way of predicting index constituents using
    logistic regression.
  \end{preface}
  \hypersetup{linkcolor=black}
  \setcounter{tocdepth}{2}
  \tableofcontents

  \listoftables

  \listoffigures
  \begin{abstract}
    The low-risk anomaly has created opportunities for arbitrage in the
    financial markets. As Baker et al. discuss in ``Benchmarks as Limits to
    Arbitrage: Understanding the Low-Volatility Anomaly,'' low-volatility
    and low-beta portfolios outperform and high-volatility and high-beta
    portfolios by a factor of several times due to benchmarking and
    lottery-preferences. The iShares MSCI USA Minimum Volatility (USMV) is
    an ETF tracking a minimum volatility index that was used to find data
    and will be used for trading arbitrage. Frazzini et al. discuss
    arbitrage opportunities by quantitative focused funds like AQR in
    ``Betting Against Beta'', and this thesis explores a more advanced type
    of index front-running as a potential arbitrage opportunity. Data was
    collected from USMV from its inception in October 2011, and from EUSA,
    the parent ETF of USMV, from the same period until December 2016.
    52-week trailing beta, 52-week trailing volatility, lagged price/book,
    and current index membership were calculated, and a regression model was
    run to quantify the relationship between current index membership and
    these four variables. In the model, a probabilities of index membership
    were calculated and an optimal cutoff was calculated to which the model
    would be 95\% accurate of its findings of a stock to be in or out of
    USMV, given the historical data. Backtesting with prior data showed with
    a model accuracy of 95\%, arbitrage opportunities of X\% could be
    collected after each rebalancing.
  \end{abstract}

\mainmatter % here the regular arabic numbering starts
\pagestyle{fancyplain} % turns page numbering back on

\chapter{thesisdown::thesis\_gitbook:
default}\label{thesisdownthesis_gitbook-default}

Placeholder

\chapter{Introduction}\label{introduction}

Placeholder

\section{Background}\label{background}

\subsection{Exchange Traded Funds
(ETFs)}\label{exchange-traded-funds-etfs}

\subsection{iShares MSCI Min Vol USA
ETF}\label{ishares-msci-min-vol-usa-etf}

\subsection{Purpose}\label{purpose}

\subsection{Logistic Regression Model}\label{logistic-regression-model}

\section{Literature Review}\label{literature-review}

\subsection{Overview of the Low-Risk
Anomaly}\label{overview-of-the-low-risk-anomaly}

\subsection{Evidence of the Low-Risk
Anomaly}\label{evidence-of-the-low-risk-anomaly}

\subsubsection{Measures of Risk}\label{measures-of-risk}

\subsubsection{1929-2015}\label{section}

\subsubsection{1968-2008}\label{section-1}

\subsubsection{Critique of the Capital Asset Pricing
Model}\label{critique-of-the-capital-asset-pricing-model}

\subsection{Possible Explanations for the Low-Risk
Anomaly}\label{possible-explanations-for-the-low-risk-anomaly}

\subsubsection{Compounding}\label{compounding}

\subsubsection{Benchmarking}\label{benchmarking}

\subsubsection{Single-Period Returns}\label{single-period-returns}

\subsubsection{Psychological and Behavioral
Factors}\label{psychological-and-behavioral-factors}

\subsubsection{Profitability and Value}\label{profitability-and-value}

\subsection{Further Decomposition of the Low-Risk Anomaly into Micro and
Macro
Effects}\label{further-decomposition-of-the-low-risk-anomaly-into-micro-and-macro-effects}

\subsection{Real-World Applications of the Low-Risk
Anomaly}\label{real-world-applications-of-the-low-risk-anomaly}

\subsubsection{Betting Against Beta
(BaB)}\label{betting-against-beta-bab}

\subsubsection{Betting Against
Correlation}\label{betting-against-correlation}

\subsubsection{Stock Price Response to Index
Rebalancing}\label{stock-price-response-to-index-rebalancing}

\chapter{Data Gathering Process}\label{data-gathering-process}

Placeholder

\section{Data Aggregation}\label{data-aggregation}

\section{Data Cleaning}\label{data-cleaning}

\subsection{Non-US Exchanges}\label{non-us-exchanges}

\subsubsection{Mislabeled Exchanges}\label{mislabeled-exchanges}

\subsubsection{Mislabeled Tickers}\label{mislabeled-tickers}

\subsection{Unrecognized Tickers}\label{unrecognized-tickers}

\subsection{Price Discrepancies}\label{price-discrepancies}

\section{Data Overview}\label{data-overview}

\section{Data Check}\label{data-check}

\subsection{Weights}\label{weights}

\subsection{Comparing ETF returns to Constructed Index
returns}\label{comparing-etf-returns-to-constructed-index-returns}

\chapter{Tables, Graphics, References, and Labels}\label{ref-labels}

\section{Change in 5 largest holdings by average weight for EUSA and
USMV}\label{change-in-5-largest-holdings-by-average-weight-for-eusa-and-usmv}

The next thing we want to see is how the top 5 largest holdings, by
average weight, in each index have changed in weighting over time. For
EUSA, the 5 largest holdings were AAPL, XOM, MSFT, GE, and JNJ. Their
change in weights are shown below.

\includegraphics{thesis_files/figure-latex/unnamed-chunk-1-1.pdf}

Shown above, for EUSA, are some very interesting findings. The weights
of the 5 companies are all very high, then suddenly all spike. Verifying
this in the data, showed that for all 5 companies, holdings dropped
significantly between 2015-07-31 and 2015-08-31. The reason for this is
not entirely clear, but the general ETF started performing poorly around
this time too. In July of 2015 the price per share was 45.20, then it
dropped to 42.60 the following month, and dropped again to 40.50 in
August 2015. Perhaps these large companies were doing poorly, and MSCI
decided to try underweighting them.
\begin{figure}[htbp]
\centering
\includegraphics{thesis_files/figure-latex/unnamed-chunk-3-1.pdf}
\caption{\label{fig:unnamed-chunk-3}For USMV, the 5 largest holdings were
VZ, T, ADP, JNJ, and MCD. Their change in weights are shown below. As we
can see below, with the exception of Verizon, the holdings generally
remain between 1 and 1.6 percent of the overall portfolio.}
\end{figure}
\section{Tables}\label{tables}

In addition to the tables that can be automatically generated from a
data frame in \textbf{R} that you saw in {[}R Markdown Basics{]} using
the \texttt{kable} function, you can also create tables using
\emph{pandoc}. (More information is available at
\url{http://pandoc.org/README.html\#tables}.) This might be useful if
you don't have values specifically stored in \textbf{R}, but you'd like
to display them in table form. Below is an example. Pay careful
attention to the alignment in the table and hyphens to create the rows
and columns.
\begin{longtable}[]{@{}ccc@{}}
\caption{\label{tab:inher} Correlation of Inheritance Factors for Parents
and Child}\tabularnewline
\toprule
\begin{minipage}[b]{0.29\columnwidth}\centering\strut
Factors\strut
\end{minipage} & \begin{minipage}[b]{0.47\columnwidth}\centering\strut
Correlation between Parents \& Child\strut
\end{minipage} & \begin{minipage}[b]{0.16\columnwidth}\centering\strut
Inherited\strut
\end{minipage}\tabularnewline
\midrule
\endfirsthead
\toprule
\begin{minipage}[b]{0.29\columnwidth}\centering\strut
Factors\strut
\end{minipage} & \begin{minipage}[b]{0.47\columnwidth}\centering\strut
Correlation between Parents \& Child\strut
\end{minipage} & \begin{minipage}[b]{0.16\columnwidth}\centering\strut
Inherited\strut
\end{minipage}\tabularnewline
\midrule
\endhead
\begin{minipage}[t]{0.29\columnwidth}\centering\strut
Education\strut
\end{minipage} & \begin{minipage}[t]{0.47\columnwidth}\centering\strut
-0.49\strut
\end{minipage} & \begin{minipage}[t]{0.16\columnwidth}\centering\strut
Yes\strut
\end{minipage}\tabularnewline
\begin{minipage}[t]{0.29\columnwidth}\centering\strut
Socio-Economic Status\strut
\end{minipage} & \begin{minipage}[t]{0.47\columnwidth}\centering\strut
0.28\strut
\end{minipage} & \begin{minipage}[t]{0.16\columnwidth}\centering\strut
Slight\strut
\end{minipage}\tabularnewline
\begin{minipage}[t]{0.29\columnwidth}\centering\strut
Income\strut
\end{minipage} & \begin{minipage}[t]{0.47\columnwidth}\centering\strut
0.08\strut
\end{minipage} & \begin{minipage}[t]{0.16\columnwidth}\centering\strut
No\strut
\end{minipage}\tabularnewline
\begin{minipage}[t]{0.29\columnwidth}\centering\strut
Family Size\strut
\end{minipage} & \begin{minipage}[t]{0.47\columnwidth}\centering\strut
0.18\strut
\end{minipage} & \begin{minipage}[t]{0.16\columnwidth}\centering\strut
Slight\strut
\end{minipage}\tabularnewline
\begin{minipage}[t]{0.29\columnwidth}\centering\strut
Occupational Prestige\strut
\end{minipage} & \begin{minipage}[t]{0.47\columnwidth}\centering\strut
0.21\strut
\end{minipage} & \begin{minipage}[t]{0.16\columnwidth}\centering\strut
Slight\strut
\end{minipage}\tabularnewline
\bottomrule
\end{longtable}
We can also create a link to the table by doing the following: Table
\ref{tab:inher}. If you go back to {[}Loading and exploring data{]} and
look at the \texttt{kable} table, we can create a reference to this max
delays table too: Table \ref{tab:maxdelays}. The addition of the
\texttt{(\textbackslash{}\#tab:inher)} option to the end of the table
caption allows us to then make a reference to Table
\texttt{\textbackslash{}@ref(tab:label)}. Note that this reference could
appear anywhere throughout the document after the table has appeared.

\clearpage

\section{Figures}\label{figures}

If your thesis has a lot of figures, \emph{R Markdown} might behave
better for you than that other word processor. One perk is that it will
automatically number the figures accordingly in each chapter. You'll
also be able to create a label for each figure, add a caption, and then
reference the figure in a way similar to what we saw with tables
earlier. If you label your figures, you can move the figures around and
\emph{R Markdown} will automatically adjust the numbering for you. No
need for you to remember! So that you don't have to get too far into
LaTeX to do this, a couple \textbf{R} functions have been created for
you to assist. You'll see their use below.

In the \textbf{R} chunk below, we will load in a picture stored as
\texttt{reed.jpg} in our main directory. We then give it the caption of
``Reed logo'', the label of ``reedlogo'', and specify that this is a
figure. Make note of the different \textbf{R} chunk options that are
given in the R Markdown file (not shown in the knitted document).
\begin{Shaded}
\begin{Highlighting}[]
\KeywordTok{include_graphics}\NormalTok{(}\DataTypeTok{path =} \StringTok{"figure/reed.jpg"}\NormalTok{)}
\end{Highlighting}
\end{Shaded}
\begin{figure}[htbp]
\centering
\includegraphics{figure/reed.jpg}
\caption{\label{fig:reedlogo}Reed logo}
\end{figure}
Here is a reference to the Reed logo: Figure \ref{fig:reedlogo}. Note
the use of the \texttt{fig:} code here. By naming the \textbf{R} chunk
that contains the figure, we can then reference that figure later as
done in the first sentence here. We can also specify the caption for the
figure via the R chunk option \texttt{fig.cap}.

\clearpage 

Below we will investigate how to save the output of an \textbf{R} plot
and label it in a way similar to that done above. Recall the
\texttt{flights} dataset from Chapter \ref{rmd-basics}. (Note that we've
shown a different way to reference a section or chapter here.) We will
next explore a bar graph with the mean flight departure delays by
airline from Portland for 2014. Note also the use of the \texttt{scale}
parameter which is discussed on the next page.
\begin{Shaded}
\begin{Highlighting}[]
\NormalTok{flights %>%}\StringTok{ }\KeywordTok{group_by}\NormalTok{(carrier) %>%}
\StringTok{  }\KeywordTok{summarize}\NormalTok{(}\DataTypeTok{mean_dep_delay =} \KeywordTok{mean}\NormalTok{(dep_delay)) %>%}
\StringTok{  }\KeywordTok{ggplot}\NormalTok{(}\KeywordTok{aes}\NormalTok{(}\DataTypeTok{x =} \NormalTok{carrier, }\DataTypeTok{y =} \NormalTok{mean_dep_delay)) +}
\StringTok{  }\KeywordTok{geom_bar}\NormalTok{(}\DataTypeTok{position =} \StringTok{"identity"}\NormalTok{, }\DataTypeTok{stat =} \StringTok{"identity"}\NormalTok{, }\DataTypeTok{fill =} \StringTok{"red"}\NormalTok{)}
\end{Highlighting}
\end{Shaded}
\begin{figure}[htbp]
\centering
\includegraphics{thesis_files/figure-latex/delaysboxplot-1.pdf}
\caption{\label{fig:delaysboxplot}Mean Delays by Airline}
\end{figure}
Here is a reference to this image: Figure \ref{fig:delaysboxplot}.

A table linking these carrier codes to airline names is available at
\url{https://github.com/ismayc/pnwflights14/blob/master/data/airlines.csv}.

\clearpage

Next, we will explore the use of the \texttt{out.extra} chunk option,
which can be used to shrink or expand an image loaded from a file by
specifying \texttt{"scale=\ "}. Here we use the mathematical graph
stored in the ``subdivision.pdf'' file.
\begin{figure}
\includegraphics[scale=0.75]{figure/subdivision} \caption{Subdiv. graph}\label{fig:subd}
\end{figure}
Here is a reference to this image: Figure \ref{fig:subd}. Note that
\texttt{echo=FALSE} is specified so that the \textbf{R} code is hidden
in the document.

\textbf{More Figure Stuff}

Lastly, we will explore how to rotate and enlarge figures using the
\texttt{out.extra} chunk option. (Currently this only works in the PDF
version of the book.)
\begin{figure}
\includegraphics[angle=180, scale=1.1]{figure/subdivision} \caption{A Larger Figure, Flipped Upside Down}\label{fig:subd2}
\end{figure}
As another example, here is a reference: Figure \ref{fig:subd2}.

\section{Footnotes and Endnotes}\label{footnotes-and-endnotes}

You might want to footnote something.\footnote{footnote text} The
footnote will be in a smaller font and placed appropriately. Endnotes
work in much the same way. More information can be found about both on
the CUS site or feel free to reach out to
\href{mailto:data@reed.edu}{\nolinkurl{data@reed.edu}}.

\section{Bibliographies}\label{bibliographies}

Of course you will need to cite things, and you will probably accumulate
an armful of sources. There are a variety of tools available for
creating a bibliography database (stored with the .bib extension). In
addition to BibTeX suggested below, you may want to consider using the
free and easy-to-use tool called Zotero. The Reed librarians have
created Zotero documentation at
\url{http://libguides.reed.edu/citation/zotero}. In addition, a tutorial
is available from Middlebury College at
\url{http://sites.middlebury.edu/zoteromiddlebury/}.

\emph{R Markdown} uses \emph{pandoc} (\url{http://pandoc.org/}) to build
its bibliographies. One nice caveat of this is that you won't have to do
a second compile to load in references as standard LaTeX requires. To
cite references in your thesis (after creating your bibliography
database), place the reference name inside square brackets and precede
it by the ``at'' symbol. For example, here's a reference to a book about
worrying: (Molina \& Borkovec, 1994). This \texttt{Molina1994} entry
appears in a file called \texttt{thesis.bib} in the \texttt{bib} folder.
This bibliography database file was created by a program called BibTeX.
You can call this file something else if you like (look at the YAML
header in the main .Rmd file) and, by default, is to placed in the
\texttt{bib} folder.

For more information about BibTeX and bibliographies, see our CUS site
(\url{http://web.reed.edu/cis/help/latex/index.html})\footnote{Reed~College
  (2007)}. There are three pages on this topic: \emph{bibtex} (which
talks about using BibTeX, at
\url{http://web.reed.edu/cis/help/latex/bibtex.html}),
\emph{bibtexstyles} (about how to find and use the bibliography style
that best suits your needs, at
\url{http://web.reed.edu/cis/help/latex/bibtexstyles.html}) and
\emph{bibman} (which covers how to make and maintain a bibliography by
hand, without BibTeX, at
\url{http://web.reed.edu/cis/help/latex/bibman.html}). The last page
will not be useful unless you have only a few sources.

If you look at the YAML header at the top of the main .Rmd file you can
see that we can specify the style of the bibliography by referencing the
appropriate csl file. You can download a variety of different style
files at \url{https://www.zotero.org/styles}. Make sure to download the
file into the csl folder.

\textbf{Tips for Bibliographies}
\begin{itemize}
\tightlist
\item
  Like with thesis formatting, the sooner you start compiling your
  bibliography for something as large as thesis, the better. Typing in
  source after source is mind-numbing enough; do you really want to do
  it for hours on end in late April? Think of it as procrastination.
\item
  The cite key (a citation's label) needs to be unique from the other
  entries.
\item
  When you have more than one author or editor, you need to separate
  each author's name by the word ``and'' e.g.
  \texttt{Author\ =\ \{Noble,\ Sam\ and\ Youngberg,\ Jessica\},}.
\item
  Bibliographies made using BibTeX (whether manually or using a manager)
  accept LaTeX markup, so you can italicize and add symbols as
  necessary.
\item
  To force capitalization in an article title or where all lowercase is
  generally used, bracket the capital letter in curly braces.
\item
  You can add a Reed Thesis citation\footnote{Noble (2002)} option. The
  best way to do this is to use the phdthesis type of citation, and use
  the optional ``type'' field to enter ``Reed thesis'' or
  ``Undergraduate thesis.''
\end{itemize}
\section{Anything else?}\label{anything-else}

If you'd like to see examples of other things in this template, please
contact the Data @ Reed team (email
\href{mailto:data@reed.edu}{\nolinkurl{data@reed.edu}}) with your
suggestions. We love to see people using \emph{R Markdown} for their
theses, and are happy to help.

\chapter*{Conclusion}\label{conclusion}
\addcontentsline{toc}{chapter}{Conclusion}

Placeholder

\chapter{The First Appendix}\label{the-first-appendix}

Placeholder

\chapter*{References}\label{references}
\addcontentsline{toc}{chapter}{References}

Placeholder

\hypertarget{refs}{}
\hypertarget{ref-Molina1994}{}
Molina, S. T., \& Borkovec, T. D. (1994). The Penn State worry
questionnaire: Psychometric properties and associated characteristics.
In G. C. L. Davey \& F. Tallis (Eds.), \emph{Worrying: Perspectives on
theory, assessment and treatment} (pp. 265--283). New York: Wiley.

\hypertarget{ref-noble2002}{}
Noble, S. G. (2002). \emph{Turning images into simple line-art}
(Undergraduate thesis). Reed College.

\hypertarget{ref-reedweb2007}{}
Reed~College. (2007, march). LaTeX your document. Retrieved from
\url{http://web.reed.edu/cis/help/LaTeX/index.html}


% Index?

\end{document}
